\documentclass[11pt,a4paper]{article}

\usepackage{marginnote}
\usepackage[top=3cm, bottom=4cm, outer=3.5cm, inner=3.5cm, heightrounded, marginparwidth=2cm, marginparsep=1cm]{geometry}

\usepackage{amsmath}
\usepackage{amsthm}

\theoremstyle{definition}
\newtheorem{thm}{Theorem}
\newtheorem{exr}{Exercise}

\setlength{\parindent}{0pt}

\usepackage{listings}
\lstset{language=Haskell}
\lstset{basicstyle=\footnotesize\ttfamily,breaklines=true}

\begin{document}

\section*{Exercises}


\subsection*{Chapter 3}
% -------------------------------------------------------------------

\marginnote{\scriptsize{pp.52-53}}[7.5mm]


% -- Ex.01 ----------------------------------------------------------

\begin{exr}
  Write a recursive function \texttt{copy :: [a] -> [a]} that copies 
  its list argument. For example, \texttt{copy [2] $\Rightarrow$ [2]}.
\end{exr}

\textbf{Solution a}
\begin{lstlisting}
  copy :: [a] -> [a]
  copy []     = []
  copy (x:xs) = x : copy xs
\end{lstlisting}

\textbf{Solution b}
\begin{lstlisting}
  copy :: [a] -> [a]
  copy = map id
\end{lstlisting}


% -- Ex.02 ----------------------------------------------------------
\vspace{5mm}

\begin{exr}
  Write a function \texttt{inverse} that takes a list of pairs and 
  swaps the pair elements. For example,
  \begin{lstlisting}
    inverse [(1,2), (3,4)] ==> [(2,1), (4,3)]
  \end{lstlisting}
\end{exr}

\textbf{Solution a}
\begin{lstlisting}
  inverse :: [a] -> [a]
  inverse []         = []
  inverse ((a,b):xs) = (b,a) : inverse xs
\end{lstlisting}

\textbf{Solution b}
\begin{lstlisting}
  inverse :: [a] -> [a]
  inverse = map (\(a, b) -> (b, a))
\end{lstlisting}

\textbf{Solution c}
\begin{lstlisting}
  import Data.Tuple
  inverse :: [a] -> [a]
  inverse = map swap
\end{lstlisting}


% -- Ex.03 ----------------------------------------------------------
\vspace{5mm}

\begin{exr}
  Write a function

  \begin{lstlisting}
    merge :: Ord a => [a] -> [a] -> [a]
  \end{lstlisting}

  which takes two sorted lists and returns a sorted list containing the 
  elements of each.
\end{exr}

\textbf{Solution}
\begin{lstlisting}
  merge :: Ord a :: [a] -> [a] -> [a]
  merge [] ys = ys
  merge (x:xs) ys = 
    let 
      smaller = [ y | y <- ys, y <= x ]
      bigger  = [ y | y <- ys, y > x ]
    in
      merge xs (smaller ++ [x] ++ bigger)
\end{lstlisting}


% -- Ex.04 ----------------------------------------------------------
\vspace{5mm}

\begin{exr}
  Write \texttt{(!!)}, a function that takes a natural number \texttt{n} 
  and a list and selects the \emph{n}th element of the list. List elements
  are indexed from 0, not 1, and since the type of the incoming number 
  does not prevent it from being out of range, the result should be 
  a \texttt{Maybe} type. For example,

  \begin{lstlisting}
    [1,2,3]!!0 ==> Just 1
    [1,2,3]!!2 ==> Just 3
    [1,2,3]!!5 ==> Nothing
  \end{lstlisting}
\end{exr}

\textbf{Solution}
\begin{lstlisting}
  (!!) :: [a] -> Int -> Maybe a
  (!!) [] _     = Nothing
  (!!) (x:xs) 0 = Just x
  (!!) (x:xs) n = (!!) xs (n-1)
\end{lstlisting}


% -- Ex.05 ----------------------------------------------------------
\vspace{5mm}

\begin{exr}
  Write a function \texttt{lookup} that takes a value and a list of pairs, 
  and returns the second element of the pair that has the value as its 
  first element. Use a Maybe type to indicate whether the lookup succeeded. 
  For example,

  \begin{lstlisting}
    lookup 5 [(1,2),(5,3)] ==> Just 3
    lookup 6 [(1,2),(5,3)] ==> Nothing
  \end{lstlisting}
\end{exr}

\textbf{Solution}
\begin{lstlisting}
  lookup :: Int -> [(Int, a)] -> Maybe a
  lookup _ [] = Nothing
  lookup n ((k,v):xs)
    | n == k    = Just v
    | otherwise = lookup n xs
\end{lstlisting}


% -- Ex.06 ----------------------------------------------------------
\vspace{5mm}

\begin{exr}
  Write a function that counts the number of times an element appears in a list.
\end{exr}

\textbf{Solution}
\begin{lstlisting}
  count :: Eq a => [a] -> a -> Int
  count []     n  = 0
  count (x:xs) n
    | x == n    = 1 + count xs n
    | otherwise = count xs n
\end{lstlisting}


% -- Ex.07 ----------------------------------------------------------
\vspace{5mm}

\begin{exr}
  Write a function that takes a value \texttt{e} and a list of values \texttt{xs}
  and removes all occurrences of \texttt{e} from \texttt{xs}.
\end{exr}

\textbf{Solution}
\begin{lstlisting}
  remove :: Eq a => [a] -> a -> [a]
  remove []     _ = []
  remove (x:xs) e
    | e == x      = remove xs e
    | otherwise   = x : remove xs e
\end{lstlisting}


% -- Ex.08 ----------------------------------------------------------
\vspace{5mm}

\begin{exr}
  Write a function 
  
  \begin{lstlisting}
    f :: [a] -> [a]
  \end{lstlisting}

  that removes alternating elements of its list argument, starting with 
  the first one. For examples, \texttt{f [1,2,3,4,5,6,7]} returns 
  \texttt{[2,4,6]}.
\end{exr}

\textbf{Solution}
\begin{lstlisting}
  alternating :: [a] -> [a]
  alternating (x:[])    = []
  alternating (x:xx:[]) = xx : []
  alternating (x:xx:xs) = xx : alternating xs
\end{lstlisting}


% -- Ex.09 ----------------------------------------------------------
\vspace{5mm}

\begin{exr}
  Write a function \texttt{extract :: [Maybe a] -> [a]} that takes a list of 
  \texttt{Maybe} values and returns the elements they contain. For example, 
  \texttt{extract [Just 3, Nothing, Just 7] = [3, 7]}.
\end{exr}

\textbf{Solution}
\begin{lstlisting}
  extract :: [Maybe a] -> [a]
  extract []             = []
  extract ((Nothing):xs) = extract xs
  extract ((Just x):xs)  = x : extract xs
\end{lstlisting}


% -- Ex.10 ----------------------------------------------------------
\vspace{5mm}

\begin{exr}
  Write a function

  \begin{lstlisting}
    f :: String -> String -> Maybe Int
  \end{lstlisting}

  that takes two strings. If the second string appears within the first, 
  it returns the index identifying where it starts. Indexes start from 0. 
  For example,

  \begin{lstlisting}
    f "abcde" "bc" ==> Just 1
    f "abcde" "fg" ==> Nothing
  \end{lstlisting}
\end{exr}

\textbf{Solution}
\begin{lstlisting}
  f :: String -> String -> Maybe Int
  f text query = search text query 0
    where tokenOf       = take (length query)
          search [] _ _ = Nothing
          search t q i  = if tokenOf t == q
                          then Just i
                          else search (tail t) q (i+1)
\end{lstlisting}


% -- Ex.11 ----------------------------------------------------------
\marginnote{\scriptsize{p.56}}[8mm]

\begin{exr}
  Write \texttt{foldrWith}, a function that behaves like \texttt{foldr} except 
  that it takes a function of three arguments and two lists.
\end{exr}

\textbf{Solution}
\begin{lstlisting}
  foldrWith :: (a -> b -> c -> a) -> a -> [b] -> [c] -> a
  foldrWith _ acc [] _ = acc
  foldrWith _ acc _ [] = acc
  foldrWith f acc (x:xs) (y:ys) = foldrWith f (f acc x y) xs ys
\end{lstlisting}


% -- Ex.12 ----------------------------------------------------------
\vspace{5mm}

\begin{exr}
  Using \texttt{foldr}, write a function \texttt{mappend} such that

  \begin{lstlisting}
    mappend f xs = concat (map f xs)
  \end{lstlisting}
\end{exr}

\textbf{Solution}
\begin{lstlisting}
  mappend :: [a] -> [a]
  mappend f xs = foldr ff [] xs
    where ff e acc = f e ++ acc
\end{lstlisting}


% -- Ex.13 --------------------------------------------------------------------
\vspace{5mm}

\begin{exr}
  Write \texttt{removeDuplicates}, a function that takes a list and 
  removes all of its duplicate elements.
\end{exr}

\textbf{Solution}
\begin{lstlisting}
  removeDuplicates :: [a] -> [a]
  removeDuplicates = foldr addUniq []
    where addUniq acc e = if e `elem` acc
                          then acc
                          else e : acc
\end{lstlisting}


% -- Ex.14 --------------------------------------------------------------------
\vspace{5mm}

\begin{exr}
  Write a recursive function that takes a value and a list of values and returns 
  \texttt{True} if the value is in the list and \texttt{False} otherwise.
\end{exr}

\textbf{Solution}
\begin{lstlisting}
  onList :: Eq a => a -> [a] -> Bool
  onList _ [] = False
  onList e (x:xs) = if e == x
                    then True
                    else onList e xs
\end{lstlisting}


% -- Ex.15 --------------------------------------------------------------------
\vspace{2mm}
\marginnote{\scriptsize{pp.59-60}}[7.5mm]

\begin{exr}
  Write a function that takes two lists, and returns a list of values that appear 
  in both lists. The function should have type 

  \begin{lstlisting}  
    intersection :: Eq a => [a] -> [a] -> [a]
  \end{lstlisting}

  (This is one way to implement the intersection operation on sets; see Chapter 8.)
\end{exr}

\textbf{Solution}
\begin{lstlisting}
  intersection :: Eq a => [a] -> [a] -> [a]                                       
  intersection xs ys = foldr inter [] xs                                          
    where inter x acc                                                             
            | x `elem` acc = acc                                                  
            | x `elem` ys  = x : acc                                              
            | otherwise    = acc  
\end{lstlisting}


% -- Ex.16 --------------------------------------------------------------------
\newpage

\begin{exr}
  Write a function that takes two lists, and returns \texttt{True} if all the 
  elements of the first list also occur in the other. The function should have 
  type \texttt{isSubset :: Eq a => [a] -> [a] -> Bool}. (This is one way to 
  determine whether one set is a subset of another; see Chapter 8.)
\end{exr}

\textbf{Solution}
\begin{lstlisting}
  isSubset :: Eq a => [a] -> [a] -> Bool
  isSubset xs ys = foldl inYs True xs
    where inYs acc x = x `elem` ys && acc
\end{lstlisting}


% -- Ex.17 --------------------------------------------------------------------
\vspace{5mm}

\begin{exr}
  Write a recursive function that determines whether a list is sorted.
\end{exr}

\textbf{Solution}
\begin{lstlisting}
  isSorted :: Ord a => [a] -> Bool
  isSorted []        = True
  isSorted (_:[])    = True
  isSorted (x:xx:xs) = if x < xx
                       then isSorted (xx:xs)
                       else False
\end{lstlisting}


% -- Ex.18 --------------------------------------------------------------------
\vspace{5mm}

\begin{exr}
  Show that the definition of \texttt{factorial} using \texttt{foldr} always 
  produces the same result as the recursive definition given in the previous 
  section.
\end{exr}

\textbf{Solution}
\begin{lstlisting}
  -- recursive definition
  factorial :: Int -> Int
  factorial 0 = 1
  factorial n = factorial (n-1) * n

  factorial 3
  = (factorial 2) * 3
  = ((factorial 1) * 2) * 3
  = (((factorial 0) * 1) * 2) * 3
  = ((1 * 1) * 2) * 3
  = (1 * 2) * 3
  = 2 * 3
  = 6

  -- foldr definition
  factorial :: Int -> Int
  factorial n = foldr (*) 1 [1..n]

  factorial 3
  = foldr (*) 1 [1, 2, 3]
  = 1 * foldr (*) 1 [2, 3]
  = 1 * (2 * foldr (*) 1 [3])
  = 1 * (2 * (3 * foldr (*) 1 []))
  = 1 * (2 * (3 * 1))
  = 1 * (2 * 3)
  = 1 * 6
  = 6
  

\end{lstlisting}


% -- Ex.19 --------------------------------------------------------------------
\vspace{5mm}

\begin{exr}
  Using recursion, define \texttt{last}, a function that takes a list and returns 
  a \texttt{Maybe} type that is \texttt{Nothing} if the list is empty.
\end{exr}

\textbf{Solution}
\begin{lstlisting}
  last :: [a] -> Maybe a
  last []     = Nothing
  last (x:[]) = Just a
  last (x:xs) = last xs
\end{lstlisting}


% -- Ex.20 --------------------------------------------------------------------
\vspace{5mm}

\begin{exr}
  Using recursion, write two functions that expect a string containing a number 
  that contains a decimal point (for example, \texttt{23.455}). The first function 
  returns the whole part of the number (i.e., the part to the left of the decimal 
  point). The second function returns the fractional part (the part to the right 
  of the decimal point).
\end{exr}

\textbf{Solution}
\begin{lstlisting}
  whole :: String -> String
  whole ""     = ""
  whole (s:ss) = if s == '.'
                then ""
                else s : (whole ss)

  decimal :: String -> String
  decimal = reverse . whole . reverse
\end{lstlisting}

% % -- Ex. --------------------------------------------------------------------
% \vspace{5mm}

% \begin{exr}
  
% \end{exr}

% \textbf{Solution}
% \begin{lstlisting}
  
% \end{lstlisting}


\end{document}