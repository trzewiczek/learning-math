\documentclass[11pt,a4paper]{article}

\usepackage{marginnote}
\usepackage[top=3cm, bottom=4cm, outer=3.5cm, inner=3.5cm, heightrounded, marginparwidth=2cm, marginparsep=1cm]{geometry}

\usepackage{amsmath}
\usepackage{amsthm}

\theoremstyle{definition}
\newtheorem{thm}{Theorem}
\newtheorem{exr}{Exercise}

\setlength{\parindent}{0pt}

\usepackage{listings}
\lstset{language=Haskell}
\lstset{basicstyle=\footnotesize\ttfamily,breaklines=true}

\begin{document}
\section*{Chapter 2. Equational Reasoning}


% -----------------------------------------------------------------------------
\vspace{5mm}\rule{\textwidth}{0.1pt}\vspace{5mm}
\marginnote{\scriptsize{p.38}}[4.5mm]

\emph{equations in Haskell are true mathematical equations}---they are not
assignment statements.


% -----------------------------------------------------------------------------
\vspace{5mm}\rule{\textwidth}{0.1pt}
\marginnote{\scriptsize{pp.38-39}}[7.5mm]

\begin{thm}[length (++)]
  Let \texttt{xs, ys :: [a]} be arbitrary lists.
  Then \texttt{length (xs ++ ys) = length xs + length ys}.
\end{thm}

\begin{thm}[length map]
  Let \texttt{xs :: [a]} be arbitrary list and \texttt{f :: a -> b} an
  arbitraty function. Then \texttt{length (map f xs) = length xs}.
\end{thm}

\begin{thm}[map (++)]
  Let \texttt{xs, ys :: [a]} be arbitrary lists and \texttt{f :: a -> b} an
  arbitrary function. Then \texttt{map f (xs ++ ys) = map f xs ++ map f ys}.
\end{thm}

\begin{thm}
  For arbitrary lists \texttt{xs, ys :: [a]},
  and arbitrary \texttt{f :: a -> b},  the following equation holds:
  \texttt{length (map f (xs ++ ys)) = length xs + length ys}.
\end{thm}

\begin{proof}
  We prove the theorem by equational reasoning, starting with the left hand side
  of the equation and transforming it into the right hand side.

  \begin{align*}
    \texttt{le}&\texttt{ngth (map f (xs ++ ys))}                           \\
      &\texttt{= length (map f xs ++ map f ys)}  & \texttt{\{ map (++) \}} \\
      &\texttt{= length (map f xs) + length (map f ys)}  & \texttt{\{ length (++) \}} \\
      &\texttt{= length xs + length ys}  & \texttt{\{ length map \}}
  \end{align*}
\end{proof}


% -----------------------------------------------------------------------------
\rule{\textwidth}{0.1pt}\vspace{5mm}
\marginnote{\scriptsize{p.44}}[4mm]

Assignments have to be understood in the context of time passing as a program executes.
To underrstand \texttt{n := n + 1} assignment, you need to talk about the old value
of the variable, and its new value. In contrast, equations are timeless, as there is
no notion of of changing the value of a variable.

% -----------------------------------------------------------------------------
\vspace{5mm}\rule{\textwidth}{0.1pt}

\newpage
\section*{Chapter 3. Recursion}


% -----------------------------------------------------------------------------
\vspace{5mm}\rule{\textwidth}{0.1pt}\vspace{5mm}
\marginnote{\scriptsize{p.48}}[4.5mm]

\begin{lstlisting}
  factorial :: Int -> Int
  factorial 0       = 1
  factorial (n + 1) = (n + 1) * factorial n
\end{lstlisting}

\vspace{5mm}

Recursive definitions consist of a collection of equations that state properties od
the function being degfined. There are a number of algebraic properties of the
factorial function, and one of them is used as the second equation of the recursive
definition. In fact, the definition doesn't consist of a set of commands to be
obeyed; it consists of a set of true equations describing the salient properties
of the function being defined.

\vspace{5mm}\rule{\textwidth}{0.1pt}\vspace{2mm}
\marginnote{\scriptsize{p.57}}[10mm]

\subsection*{Peano Arithmetic}
% -----------------------------------------------------------------------------

\begin{lstlisting}
  data Peano = Zero | Succ Peano deriving Show

  decrement :: Peano -> Peano
  decrement Zero     = Zero
  decrement (Succ a) = a

  add :: Peano -> Peano -> Peano
  add Zero     b = a
  add (Succ a) b = Succ (add a b)

  sub :: Peano -> Peano -> Peano
  sub a        Zero     = a
  sub Zero     _        = Zero
  sub (Succ a) (Succ b) = sub a b

  equals :: Peano -> Peano -> Bool
  equals Zero     Zero = True
  equals Zero     b    = False
  equals a        Zero = False
  equals (Succ a) (Succ b) = equals a b

  lt :: Peano -> Peano -> Bool
  lt a        Zero = False
  lt Zero     b    = True
  lt (Succ a) (Succ b) = lt a b
\end{lstlisting}

\end{document}
